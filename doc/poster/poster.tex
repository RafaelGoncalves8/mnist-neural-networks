\documentclass[a0,portrait]{a0poster}
% \documentclass[9pt]{article}

\usepackage[scaled]{beramono} 
\usepackage[utf8]{inputenc}
\usepackage{setspace}
\usepackage{epstopdf}
\usepackage[portuges]{babel}
\usepackage{amsfonts, amsmath, amsthm, amssymb}
\usepackage{graphicx, url, float, booktabs, datetime, multicol}
\usepackage{lipsum, cite, wrapfig}
\usepackage[font=small,labelfont=bf]{caption}
\usepackage[ruled]{algorithm2e}
\usepackage[super]{nth}
\usepackage[hang]{subfig}
\usepackage[svgnames]{xcolor} % Specify colors by their 'svgnames', for a full list of all colors available see here: http://www.latextemplates.com/svgnames-colors
\usepackage[
    breaklinks=true,    
    allbordercolors=indigo,
%    ocgcolorlinks=true,
    colorlinks=true,
    anchorcolor=indigo, 
    citecolor=indigo,
    filecolor=indigo,
    linkcolor=indigo,
    menucolor=indigo,
    runcolor=indigo,
    urlcolor=indigo,
    linktoc=all
]{hyperref}
\usepackage[ocgcolorlinks]{ocgx2}

\usepackage{pgf}
\usepackage{pgfpages}

\pgfpagesdeclarelayout{boxed}
{
  \edef\pgfpageoptionborder{0pt}
}
{
  \pgfpagesphysicalpageoptions
  {%
    logical pages=1,%
  }
  \pgfpageslogicalpageoptions{1}
  {
      border code=\pgfsetlinewidth{4pt}\color{Indigo}\pgfstroke,%
    border shrink=\pgfpageoptionborder,%
    resized width=.95\pgfphysicalwidth,%
    resized height=.95\pgfphysicalheight,%
    center=\pgfpoint{.5\pgfphysicalwidth}{.5\pgfphysicalheight}%
  }%
}

\pgfpagesuselayout{boxed}

\columnsep=100pt
\columnseprule=4pt

\usepackage{sectsty}
% \sectionfont{\color{Indigo}\Huge}
%  \sectionfont{centering\Huge}

\usepackage[most]{tcolorbox}

\tcbset{
    frame code={}
    center title,
    left=0pt,
    right=0pt,
    top=0pt,
    bottom=0pt,
    colback=Indigo!10,
    colframe=black,
    width=\dimexpr\linewidth\relax,
    enlarge left by=0mm,
    boxsep=3pt,
    arc=0pt,outer arc=0pt,
    }



\usepackage[T1]{fontenc}
\usepackage[tabular,lining]{montserrat}
\newcommand*{\headingfont}{\fontfamily{Montserrat-TOsF}\selectfont}
\usepackage{avant}
\renewcommand{\familydefault}{\sfdefault}

\setlength\parindent{5cm}

\begin{document}

    \noindent
\begin{minipage}{0.73\linewidth}
    \headingfont
\veryHuge \color{Indigo} \textbf{Fundamentos de Redes Neurais Profundas:}\\ \color{Black}
\Huge\textit{Abordagem Baseada em Redes Convolucionais.}\\[1cm] % Subtitle
\huge \textbf{Rafael Gonçalves \& Romis Attux.}\\[0.2cm] % Author(s)
    {\LARGE Faculdade de Engenharia Elétrica e de Computação - Unicamp\\
    \texttt{r186062@dac.unicamp.br, attux@dca.fee.unicamp.br}}
\end{minipage}
%
\begin{minipage}{0.27\linewidth}
    \centering
    \vspace{2cm}
    \def\svgwidth{0.6\columnwidth}
    \input{unicamp.pdf_tex}
    \break\hfill\break
\end{minipage}

\vspace{1cm} % A bit of extra whitespace between the header and poster content

\begin{multicols}{2}

    \section*{\begin{tcolorbox}
        \huge
        \headingfont
        \centering
        Introdução
    \end{tcolorbox}}
    {\Large

    Redes neurais artificiais são sistemas de computação não lineares e adaptativos originalmente inspirados nas redes neurais biológicas presentes no sistema nervoso dos animais.
    Especialmente com o advento de redes neurais profundas e o conceito de aprendizado profundo, este se tornou um importante paradigma dentro do campo de aprendizado de máquina e é amplamente utilizado para resolver uma variedade de problemas atuais.

    Neste contexto, esta pesquisa buscou estudar teoricamente redes neurais profundas baseado em um livro recente e representativo ~\cite{Goodfellow16} e posteriormente aplicar um modelo específico de rede neural -- a saber uma rede convolucional -- ao problema conhecido de reconhecimento de dígitos escritos à mão utilizando a base de dados MNIST ~\cite{mnist}.

    }
    \section*{\begin{tcolorbox}
        \huge
        \headingfont
        \centering
        Discussões e Resultados
    \end{tcolorbox}}
    {\Large
    \lipsum[2]

    }
    \section*{\begin{tcolorbox}
        \huge
        \headingfont
        \centering
        Conclusões
    \end{tcolorbox}}
    {\Large
    \lipsum[3]

    }
    \section*{\begin{tcolorbox}
        \huge
        \headingfont
        \centering
        Agradecimentos
    \end{tcolorbox}}
    {\Large

    O estudante gostaria de expressar seu agradecimento ao programa PIBIC/CNPq/Unicamp pelo auxílio financeiro e em especial ao Prof. Romis Attux por todo o incentivo e apoio durante o desenvolvimento da pesquisa.

    }

    \rule{.33\linewidth}{2pt}

{\large
\begin{thebibliography}{9}
    \bibitem{Goodfellow16}
        I. Goodfellow, Y. Bengio, A. Courville.
        \textit{Deep Learning}.
        MIT Press, 2016.
    \bibitem{mnist}
        Y. LeCun.
        \textit{The MNIST Database of Handwritten Digits}.
        \url{http://yann.lecun.com/exdb/mnist}.
        (acessado em 07/07/2019).
    \bibitem{github}
        R. Gonçalves.
        \textit{mnist\_nn}.
        \url{https://github.com/RafaelGoncalves8/mnist_nn}
        (acessado em 20/07/2019).
    \bibitem{nnsvg}
        LeNail.
        \textit{NN-SVG: Publication-Ready Neural Network Architecture Schematics}.
        \url{http://alexlenail.me/NN-SVG/}.
        Journal of Open Source Software, 2019.
        (acessado em 20/07/2019).
\end{thebibliography}
}
\end{multicols}

\end{document}
